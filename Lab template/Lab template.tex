%%%%%%%%%%%%%%%%%%%%%%%%%%%%%%%%%%%%%%%%%
% University/School Laboratory Report
% LaTeX Template
% Version 4.0 (March 21, 2022)
%
% This template originates from:
% https://www.LaTeXTemplates.com
%
% Authors:
% Vel (vel@latextemplates.com)
% Linux and Unix Users Group at Virginia Tech Wiki
%
% License:
% CC BY-NC-SA 4.0 (https://creativecommons.org/licenses/by-nc-sa/4.0/)
%
%%%%%%%%%%%%%%%%%%%%%%%%%%%%%%%%%%%%%%%%%

%----------------------------------------------------------------------------------------
%	PACKAGES AND DOCUMENT CONFIGURATIONS
%----------------------------------------------------------------------------------------

\documentclass[
	letterpaper, % Paper size, specify a4paper (A4) or letterpaper (US letter)
	10pt, % Default font size, specify 10pt, 11pt or 12pt]
]{labby boi}

\newcommand{\dd}[0]{\text{d}}
\usepackage{multicol}

%----------------------------------------------------------------------------------------
%	REPORT INFORMATION
%----------------------------------------------------------------------------------------

\title{*Title*
} % Report title

\author{*your name*} % name

\date{*Date*} % Date of the report

%----------------------------------------------------------------------------------------

\begin{document}

\maketitle % Insert the title, author and date using the information specified above

\begin{center}
	\begin{tabular}{l r}
		Date Performed: & *Date* \\ % Date the experiment was performed
		TA name: & *TA name* \\ %TA name
		Instructor: & *professor name* % Instructor/supervisor
	\end{tabular}
\end{center}

% If you need to include an abstract, uncomment the lines below
%\begin{abstract}
%	Abstract text
%\end{abstract}

%----------------------------------------------------------------------------------------
%	OBJECTIVE
%----------------------------------------------------------------------------------------

\vspace{5in}
\section{Introduction}
*Theory/concepts behind lab, and explain equations being experimentally tested*
\subsection{Objective}
*Main objective(s) of lab and how (in one sentence) it will be accomplished*\\
\textbf{Hypotheses:}\\
*Hypotheses (expected results/trends)*
\subsection{Relevant Equations}
*Relevant equations*
\subsection{Testing set-up}
*Testing set-ups*

\begin{multicols}{2}
\subsubsection{First test}
*Test one*
    \begin{enumerate}
    \item *procedure*
\end{enumerate}

\subsubsection{Second test}
*test two*
\begin{enumerate}
    \item *procedure*
\end{enumerate}

\subsubsection{Third test}
*test three*
    \begin{enumerate}
        \item *procedure*
    \end{enumerate}
\end{multicols}

% experimental set up

% \begin{figure}[H]
%     \centering
%     \includegraphics[width=0.8\textwidth]{*picture*}
%     \caption{*caption*}
% \end{figure}

\section{Experimental Data}

\subsection{First test}
\subsubsection{Tables and graphs}

%Two different figures side by side

%\begin{figure}[H]
   % \centering
    %\begin{minipage}{0.45\textwidth}
        %\centering
        %\includegraphics[width=0.9\textwidth]{fig 1} % first figure itself
        %\caption{caption 1}
   % \end{minipage}\hfill
%    \begin{minipage}{0.45\textwidth}
   %     \centering
      %  \includegraphics[width=0.9\textwidth]{fig 2} % second figure itself
        %\caption{caption 2}
  %  \end{minipage}
%\end{figure}

% \begin{table}[H]
% \centering
%     \begin{tabular}{|c|c|c|c|}
%     \hline
%      & Parameter one & Parameter two & Parameter three\\
%     \hline
%     Trial one & val 11  & val 12  & val 13 \\
%     Trial two & val 21  & val 22  & val 23 \\
%     Trial three & val 31  & val 32 & val 33 \\
%     \hline
% \end{tabular}
% \caption{*caption*}
% \end{table}

\subsubsection{Error analysis}
*error*

\subsubsection{Comments on data}
*comments on data*

\subsection{Second Test}
\subsubsection{Tables and graphs}

%\begin{figure}[H]
   % \centering
    %\begin{minipage}{0.45\textwidth}
        %\centering
        %\includegraphics[width=0.9\textwidth]{fig 1} % first figure itself
        %\caption{caption 1}
   % \end{minipage}\hfill
%    \begin{minipage}{0.45\textwidth}
   %     \centering
      %  \includegraphics[width=0.9\textwidth]{fig 2} % second figure itself
        %\caption{caption 2}
  %  \end{minipage}
%\end{figure}

% \begin{table}[H]
% \centering
%     \begin{tabular}{|c|c|c|c|}
%     \hline
%      & Parameter one & Parameter two & Parameter three\\
%     \hline
%     Trial one & val 11  & val 12  & val 13 \\
%     Trial two & val 21  & val 22  & val 23 \\
%     Trial three & val 31  & val 32 & val 33 \\
%     \hline
% \end{tabular}
% \caption{*caption*}
% \end{table}

\subsubsection{Error analysis}
*error*

\subsubsection{Comments on data}
*comments*

\section{Conclusions}
*State main findings of lab (were hypotheses supported or not?)*\\
*Discuss if trends seen were or were not expected, based on theory behind the lab*\\
*State final experimental values and errors for parameters of interest*\\
*Discuss if accepted value falls within your error range, for estimating parameters*\\
*Mention any observed outliers in data, and justify their removal, if removed during data
analysis*\\
*State sources of error (from human error and inherent in lab setup equipment), and
suggest potential solutions/fixes*

\subsection{Discussion Questions}
*add discussion questions from last slide of lab*

\subsection{Proof}

%selfies of you doing the experiment

% \begin{figure}[H]
%     \centering
%     \includegraphics[width=0.8\textwidth]{*picture*}
%     \caption{*caption*}
% \end{figure}

\end{document}
